\begin{comment}
	%%%%Overview%%%%
	- fundamental problem in CV
		- much work over previous decades
	- goal: find specific category in a given image
		- of interest by itself. Localise and classification
		- also a pre-requisite step to higher-level vision tasks
			- activity & event recognition, scene understanding, + find more
	- difficulties 
		- inter-class differences small
		- intra-class variations large
			- shape, pose, colour, texture, background, differences in illumination/viewpoint etc between images
	- GPU/deep learning advances have meant that performance is starting to become satisfactory for real-world use in scenarios requiring high precision and accuracy
		- autonomous vehicles, military, medicinal
	- much discussion of AI taking labor jobs
		- find articles and specific examples
			- elon musk, bill gates, etc (leaders of tech)
		- object detection needed
		- improvements still to be made before being completely realised
	- next section, overview of object detection definition, key task and challenges within. Also SOTA related work
\end{comment}


Object detection is a fundamental area of computer vision that has had a great amount of research over the past decades. The general goal of object detection is to find a specific object in an image. The specific object is typically from within a pre-defined list of categories that are of interest for a given use case. Object detection generally consists of two larger tasks; localisation and classification. It is assumed that the objects of interest are not already located in the image and as objects can vary in number of pixels depending on factors such as distance and scale, objects must be both localised in an image and classified accurately. Localisation is typically done by with a bounding-box indicating where a given object is in the image. However, other methods such as objects' centres and closed boundaries can also be used. Not only is object detection an important task in localising and classifying, it is also a necessary earlier step in larger computer vision pipelines. For example, object detection is needed within the tasks such as activity and event recognition, scene understanding, and robotic picking.

Object detection is a challenging problem due to both some large scale issues and minute differences. Firstly, there is the challenge of differentiating objects between classes. Depending on the problem at hand the sheer number of potential categories present can be into the thousands or tens of thousand. On top of this separate object categories can be both very different in appearance, for example an apple and an aeroplane, but separate categories can also be similar in appearance, such as dogs and wolves.

Current state-of-the-art within object detection is also within the realm of deep learning with \glspl{cnn}. This is exemplified with almost all entries in benchmark challenges such as \gls{pascalvoc} \cite{pascalvoc2012}, ImageNet \cite{imagenet}, and \gls{mscoco} \cite{mscoco} consisting of \gls{cnn} based approaches. However, improvements are still needed before object detection can be used in real-world scenarios that require a high level of precision, accuracy, and performance. 

\section{Initial Problem Statement}

\begin{comment}
	- what are specific problems within object detection?
	- 
\end{comment}

An initial problem statement can be formed as follows: \\

\textit{What are the specific challenges within object detection?} \\ \\
Based upon this, \chapterref{problemanalysis} will outline these challenges. On top of this, related work into object detection, both current state-of-the-art and also classic methods, will be researched.
