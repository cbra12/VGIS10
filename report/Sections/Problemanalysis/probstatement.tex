\section{Problem Statement}
As outlined in the introduction, the general goal of object detection is to find objects in an image based upon pre-defined object categories. As mentioned in \sectionref{probchallenges}, the main challenges can be within object detection can be defined in two groups as robust-related and computational complexity and scalability-related as per \cite{zhang}. The robust-related challenges are with respect to variations in the objects, this can include colour, texture, shape and size. The other challenge in this group is variations in the images which can differ in terms of lighting, viewpoint and quality. Both object and image variations can occur intra- and inter-class. These robust-related challenges lead into the computational-complexity and scalability-related. As object detection can be a quite difficult task the choice of model must be sufficient to capture such large variations. Additionally, this puts requirements on the quantity and quality of the data needed to train such a model. 
Based on the the works covered in \sectionref{related}, current leading methods are \gls{cnn}-based. These methods are the current leaders in choice for tackling the robust-related challenges. Additionally through the use of high-quality datasets such as \gls{pascalvoc} and \gls{mscoco} research within object detection has grown considerably within recent years. However, there is still number of challenges present in relation to both object and image variations that many not have been addressed properly yet. Many leading methods find smaller objects challenging to detect. Additionally variations in the quality of the image is an area which not many have addressed.
Despite \gls{cnn}-based methods being the current state-of-the-art and are becoming increasingly more complex they may yet find it difficult to generalise across many of the robust-related challenges.

Therefore, the following questions will be investigated in this work:

\begin{itemize}
\item \textit{How can specific robust-related challenges be addressed in \gls{cnn}-based object detector}?
\end{itemize}