This work addressed the problem of robust-related challenges in object detection using an ensemble of \glspl{rfcn}. Using guidelines for building an ensemble system it was determined to research strategies in data sampling and selection to reduce the variance of individual \gls{rfcn} ensemble members. The sampling was done with respect to variations in object resolution and image quality. Two combination strategies were analysed with respect to combing detection results of the differently data-sampled ensemble members. \gls{rpn} proposals were used to estimate potential object size and a deep IQA models were used to measure the subjective image quality. It was found that the overall performance could be increased by 0.56\% \gls{ap} in comparison to a baseline \gls{rfcn}. 

A suitable training procedure that aided in the combination of \glspl{rfcn} was also presented. Using a multi-step training procedure for the networks, ensemble members could be given the same object proposals inputs so that their individual detections were able to be combined. 

It was also found that it is not necessary for an ensemble member to be an expert for its given factor compared to the baseline model. The image quality members performed 3-4\% worse than the baseline model. However, when combined appropriately with both object resolution members and the baseline model the overall performance was increased.

The work determined that there are indications that a robust-related ensemble can aid in object detection. A number of items need to addressed, as mentioned in previous chapter, such as the heavily skewed data present for both object resolution and image quality distortions. 