This work addressed the problem of robust-related challenges in object detection. As \gls{cnn}-based ensemble object detection methods are the current leader on multiple benchmarks the decision was made to analysis this further. This lead to the following problem statement:

\begin{itemize}
\item \textit{How can specific robust-related challenges be addressed in \gls{cnn}-based object detector with the aid of ensemble methods}?
\end{itemize}

A technical analysis of object detection using \glspl{cnn} lead to the choice of \gls{rfcn} with ResNet-101 as the backbone architecture. An analysis of ensemble methods aided design choices towards building a system towards data sampling and selection to reduce the variance of individual \gls{rfcn} ensemble members. The sampling was done with respect to the robust-related variations in object resolution and image quality. Two combination strategies were analysed to combining detection results of the differently data-sampled ensemble members. \gls{rpn} proposals were used to estimate potential object size and a deep IQA models were used to measure the subjective image quality. It was found that the overall performance could be increased by 0.56\% \gls{ap} in comparison to a baseline \gls{rfcn} if the detections were combined appropriately. 
\\\\
A suitable training procedure that aided in the combination of \glspl{rfcn} was also presented. Using multi-step training for the networks, ensemble members could be given the same object proposals inputs so that their individual detections were able to be combined. 
\\\\
It was also found that it is not necessary for an ensemble member to be an expert for its given factor compared to the baseline model. The image quality members performed 3-4\% worse than the baseline model. However, when combined appropriately with both object resolution members and the baseline model the overall performance was increased.
\\\\
The work determined that there are indications that a robust-related ensemble can aid in object detection. A number of items need to addressed, as mentioned in previous chapter, such as the heavily skewed data present for both object resolution and image quality distortions. 