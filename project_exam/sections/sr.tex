\section{Super Resolution}
% motivation for creating this theme
\begin{frame}{Super Resolution}{}
    \begin{block}{Increase spatial resolution to create a more detailed image.}

        \begin{figure}[H]
            \centering
            \begin{subfigure}[b]{0.25\textwidth}
                \includegraphics[width=\textwidth]{figs//butterfly_low.jpg}
                \caption*{\scriptsize LR image}
            \end{subfigure}
            \quad
            \begin{subfigure}[b]{0.25\textwidth}
                \includegraphics[width=\textwidth]{figs/butterfly_high.jpg}
                \caption*{\scriptsize HR image}
            \end{subfigure}
        \end{figure}

        \begin{itemize}
            \item Direct solution: Increase pixel density of the image sensor.
            \item Super-Resolution: Transform one or more LR images into a HR image.
        \end{itemize}


    \end{block}
\end{frame}




\begin{frame}{Super Resolution}{}
    \begin{block}{Pros (P) and cons (C) of increasing sensor resolution vs. performing Super Resolution}
        ~\\
        \begin{columns}
            \begin{column}{0.5\textwidth}
                \textbf{Increasing sensor resolution}
                \begin{itemize}
                    \item \textbf{P} - Actual increase of resolution/details.
                    \item \textbf{C} - Expensive and cumbersome.
                    \item \textbf{C} - Higher requirements to storage capacity.
                    \item \textbf{C} - Can introduce noise and blur.
                \end{itemize}
            \end{column}
            \begin{column}{0.5\textwidth}
                \textbf{Performing SR}
                \begin{itemize}
                    \item \textbf{P} - A generic cost effective solution.
                    \item \textbf{P} - Can be used only when needed to limit storage needs.
                    \item \textbf{C} - Enhancement is a "guess".
                    \item \textbf{C} - SoTA methods are still limited.
                \end{itemize}

            \end{column}
        \end{columns}




    \end{block}
\end{frame}




\begin{frame}{Super Resolution}{}


    \begin{block}{Super-Resolution methods:}

        \begin{itemize}
            \item Multiple image methods, based on sub-pixel shifts between LR-images.
            \item Single image methods, typically learning based.
        \end{itemize}

        \centering
        \begin{columns}
            \begin{column}{0.5\textwidth}
                \textbf{SoTA Super-Resolution:} Deep Learning, scientific computing accelerated using GPUs.
            \end{column}
            \begin{column}{0.5\textwidth}

                \includegraphics[width=0.5\textwidth]{figs/tesla.jpg}\footnotemark
            \end{column}
        \end{columns}
        \footnotetext{Image source: http://www.nvidia.com/object/tesla-workstations.html}
    \end{block}
\end{frame}

